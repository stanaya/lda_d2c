\documentclass[uplatex]{jsarticle}
\和暦
\usepackage[top=25truemm,bottom=25truemm,left=25truemm,right=25truemm]{geometry}
\usepackage[dvipdfmx]{graphicx}
\usepackage{color}
\usepackage{listings}

\definecolor{dkgreen}{rgb}{0,0.6,0}
\definecolor{gray}{rgb}{0.5,0.5,0.5}
\definecolor{mauve}{rgb}{0.58,0,0.82}

\lstset{frame=tb,
  language=Python,
  aboveskip=3mm,
  belowskip=3mm,
  showstringspaces=false,
  columns=flexible,
  basicstyle={\small\ttfamily},
  numbers=none,
  numberstyle=\tiny\color{gray},
  keywordstyle=\color{blue},
  commentstyle=\color{dkgreen},
  stringstyle=\color{mauve},
  breaklines=true,
  breakatwhitespace=true,
  tabsize=3
}
\begin{document}

\title{\huge D系クエリとAAのLDAについて}
\author{木村 健}
\date{\today}
\maketitle

\section{このドキュメントについて}
このドキュメントではD系(ドコモ)クエリのLDA(木村は未達)と
AA(All About)記事のLDA(今絶賛この文章を書きながら回している)
について、作業した内容の共有と、その時生成されたデータや使ったプログラム
について説明を加える。

\section{D系クエリLDA}
まず、D系クエリの膨大なデータの蓄積がある。
これはApache Hiveのデータ構造になっていて、スキーマを
構成するフィールドにクエリ文字列と時間、個人識別のidが含まれている。
解析するためにはまずこれらのデータを単一のCSVに落とすことが
考えられるが、現状2ヶ月ほどのデータについてこれを試みたところ、
おそらく600GBくらいのデータになることが試算と実験でわかった。

このためCSVをインメモリで処理できる限界量までCSVを膨らませて
解析するか、そもそもCSVファイルで解析することを諦めて、
Hive上で最終的にLDA解析に必要なデータをデータベースの
データとして収集するか、どちらかを選択する必要があり、木村は
後者の手法(データベース上でHiveで操作)を選び作業を
途中まで進めた。

まずHiveで形態素解析をするApache Hive UDF(User Defined Function)として
mecabを使ったものを作る必要性を感じ、

\begin{quote}

https://github.com/kazuhira-r/kuromoji-with-mecab-neologd-buildscript

\end{quote}


から派生した新しいレポジトリを作成し、kuromojiを経由せずmecabを直接Java APIで
制御するHive UDFを作った。

\begin{quote}

https://github.com/kimrin/spark-hive-udf-mecab

\end{quote}

本レポジトリの一連のスクリプトは、

mecabをインストールし、neologdをインストールし、
次にHive UDFを含んだJavaプログラムをコンパイルして二つのJarに
まとめる。

使用側では、HiveをHueのインタフェース上などから、
ADDJAR してTEMPORARY FUNCTIONを作り、
これをSELECT文のフィールドに対して関数のように使う。
今のところ表層を分かち書きするsurface関数だけが提供されている。

\begin{verbatim}
ADD JAR MeCab.jar;
ADD JAR spark-hive-udf_2.10-0.1.0.jar;
CREATE TEMPORARY FUNCTION surface AS 'com.ardentex.spark.hiveudf.MecabSurface';
SELECT keyword, surface(keyword) FROM full_query WHERE ver='2018-06-01' AND sub_ver='00-00-00';
\end{verbatim}

ただ、調査の結果性能としてはD系クエリ5日分くらいでmecabのインスタンス破壊が観測され、回避策を入れたが、
結局のところmecabの対クエリ耐性からこれ以上をMapReduceやTezで実行すると途中で落ちてしまう現象に遭遇した。

結果的にこれらの危ない日付を回避してデータベースのテーブルを構成すれば良いのだが、そこまで
行かなかったのと、LDAのために必要なTF(term frequencies)をHive SQLでどのように順に集計していけばいいのか
まだ学習途中で途中までしか進められなかった。

少しspark-hive-udf-mecabに力を入れすぎた感がある。のちの調査でHive Mallに含まれているkuromojiでは
Hive Mallを導入するだけである程度の性能のKuromoji解析機がHive UDFとして使えることがわかり、
事実上spark-hive-udf-mecabは遺跡となった。

結果、LDA に必要な擬似SVMデータを得られないまま今日に至っている・・・

棚谷さんが3日分のクエリーについて結果を出しておられるそうです・・・

\section{AAのLDAについて}
AAからほぼHTMLママのCSVファイルを頂いた(zipで)。
これをPython3のbeautiful soupで解析して、
手頃な扱いのできるJSONファイルに直したところ、27GBくらいの大きさになった。
(この作業は木村が実施した)。

このJSONは次のような構成になっている。
\begin{verbatim}
{"記事番号": {
                     "記事番号枝番": {
                                                 "articles": [
                                                        {"H3": "見出し",
                                                         "text": ["テキスト1", "テキスト2", ..., "テキスト_n"]
                                                        }, {...}, {...}, ...
                                                 ]
                      }
}
\end{verbatim}

基本的にH3タグが見出しなのだが、一部fontタグで代用されているものがあった。
それらについては単純に抜き出すのではなく、bタグで囲ってあるものだけを
H3として抽出した。

またtextについては連続しているものが途中で物理的に切断されている
ものがある。このため文の区切りとしては不適切な切り方も多く存在する。

\subsection{LDAやってみた}
上記のJSONデータを使ってLDAの元データ(SVMファイル)を作ってみた。
ベースとしてja.text8のプログラムを使った。
ドキュメントとしてはH3の全てを分かち書きしたものと、textを連結した一つのテキストとして
分かち書きしたものをデータとして使った。

結果、かなり大きなsvmファイルができた。

documents = 393007, lexicon = 835732, nwords = 38600153.

今回は時間の関係上N=700とやや小さい値で検証した。
topicsも100とやや小さめである。
それでもm4.4xlargeのマシン(RAM 64GB)で17時間ほど掛かった。

時間があれば、topics=2000, N=1000くらいのタスクをやってみたい。

あと付記として、thetaをsaveするとき巨大なnp.zerosを実行するようで、
thetaをsaveする途中でメモリエラーで落ちてしまう現象が出た。
単純にthetaのセーブをコメントアウトして(現在のところ使わない)、
対処した。

結果のExcelファイルについては別箇添付する。
N=700ではあるが、AAコーパスの品質の良さが幸いしてか、
妥当な分類解となっている。

N=700時のperplexity=9466である。




\begin{thebibliography}{99}
  \bibitem{murphy} Kevin P. Murphy: Machine Learning: A Probabilistic Perspective (Adaptive Computation and Machine Learning series) The MIT Press, 2012.
  \bibitem{blei} Blei, David M. and Ng, Andrew Y. and Jordan, Michael I.: Latent Dirichlet Allocation,
J. Mach. Learn. Res. 3/1, volume 3, 2003.
\end{thebibliography}
\end{document}
